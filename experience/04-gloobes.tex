% !TEX encoding = UTF-8
\cvExperienceHeader{Gloobes}{2015 -- 2020}{La Plata, Buenos Aires, Argentina}

\cvRoleProject{Fullstack Developer}{Plataforma Ecommerce multiregional}{}
\begin{itemize}[itemsep=1pt, topsep=4pt, leftmargin=15pt]
	
	\item \textbf{Arquitectura de API REST:} Diseñé y desarrollé una API robusta para comercio electrónico utilizando \textbf{PHP 7} y \textbf{Phalcon 2}, gestionando flujos complejos de carrito de compras, procesamiento de pedidos y control de inventario.
	
	\item \textbf{Estrategia de Persistencia Híbrida:} Implementé una arquitectura de datos dual utilizando \textbf{MySQL} para la integridad de transacciones y \textbf{MongoDB} para el catálogo de productos, optimizando la flexibilidad y velocidad de respuesta en búsquedas de gran escala.
	
	\item \textbf{Integraciones Fintech Internacionales:} Realicé la integración de múltiples pasarelas de pago en \textbf{Argentina, Uruguay, México, Chile, Colombia, Estados Unidos, Portugal y Paraguay}, consumiendo servicios web mediante protocolos \textbf{SOAP} y \textbf{REST} para garantizar el pago de las ordenes de compra de los usuarios.
	
	\item \textbf{Logística:} Implementé la lógica de integración con diversos proveedores de envíos de \textbf{Argentina, Uruguay, México, Chile, Colombia, Estados Unidos, Portugal y Paraguay}, automatizando el cálculo de costos, tracking y logística mediante la conexión con sus respectivos \textbf{Web Services SOAP y/p REST}.
	
	\item \textbf{Escalabilidad Regional:} Adapté el sistema para manejar múltiples monedas, proveedores de pagos y proveedores logísticos locales, asegurando una experiencia de usuario consistente en diversos mercados.
	
	\item \textbf{Implementación de Frontend White Label:} Desarrollé un sistema de interfaz modular y personalizable para los distintos partners de la plataforma utilizando \textbf{PHP 7, Phalcon 2, JavaScript y CSS}, permitiendo el despliegue dinámico de múltiples tiendas bajo una misma base de código.
	
\end{itemize}

\vspace{10pt}

\cvRoleProject{Developer}{Emisión de comprobantes fiscales electrónicos}{}

\begin{itemize}
	\item \textbf{Desarrollo de una librería de facturación electrónica:} Diseñé e implementé una librería en \textbf{PHP 7} que funciona como una capa de abstracción para la integración con diversos proveedores de facturación electrónica. Esta solución estandarizó el envío de comprobantes fiscales mediante la normalización de datos para distintos Web Services gubernamentales y privados.
\end{itemize}

\vspace{10pt}

\cvRoleProject{Developer}{Integración de Punto de Venta Web (POS) y Hardware Fiscal}{}

\begin{itemize}[itemsep=1pt, topsep=4pt, leftmargin=15pt]
	
	\item \textbf{Desarrollo de Middleware de Comunicación:} Diseñé e implementé un servicio en \textbf{C\#} que actúa como capa de abstracción entre aplicaciones web y hardware local, permitiendo la comunicación de puntos de venta con impresoras fiscales mediante una \textbf{API de WebSockets}.
	\item \textbf{Integración de Librerías de Bajo Nivel:} Desarrollé un adaptador robusto para el consumo de librerías dinámicas (\textbf{DLL}) de diversos fabricantes de impresoras, estandarizando los comandos de impresión.
	\item \textbf{Estandarización de Protocolos:} Definí un protocolo de mensajes JSON sobre WebSockets para facilitar la integración de cualquier frontend web con periféricos locales, logrando una solución escalable y multiplataforma.
	
\end{itemize}

\vspace{10pt}
