% !TEX encoding = UTF-8

\cvExperienceHeader{Geopagos}{2020 -- Actualidad}{Ciudad Autónoma de Buenos Aires, Argentina}


\cvRoleProject{Product Owner Specialist}{Liquidación de impuestos y pagos a comercios}{2025 - Actualidad}

\begin{itemize}[itemsep=1pt, topsep=4pt, leftmargin=15pt]
	
	\item \textbf{Estrategia y gestión de backlog:} Lideré la priorización y refinamiento del backlog de producto y tecnología, logrando un equilibrio crítico entre los objetivos de negocio, la resolución de deuda técnica y la capacidad operativa del equipo.
	
	\item\textbf{ Diseño de roadmap y modernización arquitectónica:} Definí la visión estratégica para la evolución de la plataforma, diseñando una arquitectura orientada a servicios para el reemplazo progresivo de flujos legacy, priorizando métricas de \textbf{escalabilidad, resiliencia y mantenibilidad}.
	
	\item \textbf{Liderazgo de soluciones cross-team:} Coordiné la alineación de requerimientos complejos con líderes de producto y técnica de otros equipos, asegurando la consistencia funcional y técnica de los servicios transversales a nivel plataforma.
	
	\item \textbf{Gestión de partners y stakeholders:} Actué como enlace técnico y funcional principal con proveedores y socios externos, facilitando el refinamiento de requerimientos y liderando la evaluación, integración y consumo de servicios estratégicos del dominio.
	
	\item \textbf{Observabilidade}: Diseñé e implementé dashboards de control en herramientas de \textbf{APM}, estableciendo KPIs basados en métricas de performance y negocio (salud de flujos, volumen transaccional, percentiles y tiempos de respuesta) para el monitoreo proactivo de flujos críticos.
	
	\item \textbf{Análisis técnico funcional:} Aporté mi expertise como Technical Leader para validar la viabilidad de iniciativas, estimar esfuerzos y liderar el análisis de requerimientos no funcionales, incluyendo normativas impositivas internacionales, seguridad y performance.
	
\end{itemize}

\vspace{12pt} % Espacio entre proyectos dentro de la misma empresa

\cvRoleProject{Technical Lead}{Liquidación de impuestos y pagos a comercios}{2021 - 2022}

\begin{itemize}[itemsep=1pt, topsep=4pt, leftmargin=15pt]
	
	\item \textbf{Modernización y migración de arquitectura:} Lideré la transición tecnológica del ecosistema de liquidaciones, encabezando el diseño y construcción de una nueva arquitectura de microservicios basada en \textbf{Java} y \textbf{Spring Boot}. Este proceso permitió desacoplar flujos legacy críticos, mejorando la \textbf{escalabilidad} y \textbf{mantenibilidad} del sistema.
	
	
	\item \textbf{Refinamiento funcional y normativo:} Participé en el análisis estratégico de requerimientos asociados a normativas impositivas y de pagos a comercios multiregionales, traduciendo regulaciones de diversos países en especificaciones técnicas.
	
	\item \textbf{Gestión de riesgos y deuda técnica:} Lideré procesos de visibilización de deuda técnica mediante métricas de calidad y observabilidad, traduciendo hallazgos técnicos en riesgos de negocio comprensibles para \textbf{stakeholders}.
	
	
\end{itemize}

\vspace{12pt} % Espacio entre proyectos dentro de la misma empresa


\cvRoleProject{Senior backend developer}{Liquidación de impuestos y pagos a comercios}{2021 - 2022}

\begin{itemize}[itemsep=1pt, topsep=4pt, leftmargin=15pt]
	
	\item \textbf{Mantenimiento evolutivo de la solución:} Participación activa en el diseño, implementación y optimición de servicios backend críticos para la liquidación de impuestos y pagos a comercios, desarrollando motores de cálculo financiero tanto en tiempo real (\textbf{Online}) como para procesamiento masivo (\textbf{Batch}), asegurando la precisión, trazabilidad del flujo de fondos y el cumplimiento de calendarios de pago.
	
	\item \textbf{Liderazgo técnico y calidad de código:} Participé, junto con el Tech Lead, en los procesos de \textbf{Code Review} y \textbf{mentoría técnica} para desarrolladores del equipo, promoviendo estándares de diseño de software y el uso de mejores prácticas arquitectónicas.
	
\end{itemize}

\vspace{12pt} % Espacio entre proyectos dentro de la misma empresa