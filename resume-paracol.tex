\documentclass[10pt,a4paper]{article}

% Paquetes fundamentales
\usepackage[utf8]{inputenc} % Allows inputting special characters directly
\usepackage[T1]{fontenc}    % Better font encoding for accented characters
\usepackage[spanish]{babel} % Handles Spanish hyphenation and terminology
\usepackage[margin=0cm, top=0cm, bottom=0.5cm, left=0cm, right=0.5cm]{geometry} % Margenes del documento
\usepackage{enumitem}              % Para listas personalizadas
\usepackage{hyperref}              % Para enlaces a LinkedIn o Email
\usepackage{titlesec}              % Para dar formato a las secciones
\usepackage{xcolor}                % Para un toque de color sutil
\usepackage[most]{tcolorbox}	% Background colors
\usepackage{graphicx} 	% Para la foto
\usepackage{paracol} % El corazón del nuevo layout
\usepackage{lipsum} % Lorem ipsum text generator
\usepackage{multicol} % multicolum package
\usepackage{etoolbox}
\usepackage{custom} % Mi paquete de estilos custom

% Definición de colores
\definecolor{primary}{RGB}{33, 47, 61}
\definecolor{accent}{RGB}{52, 152, 219}
\definecolor{sidebarGray}{RGB}{242, 244, 244}

% --- CONFIGURACIÓN DE PARACOL ---
\columnratio{0.32} % 32% para el sidebar, el resto para el contenido
\setlength{\columnsep}{1cm} % Espacio entre columnas

% --- CONFIGURACIÓN GLOBAL ---
\setlength{\parindent}{0pt}
\pagestyle{empty}

% Formato de secciones
\titleformat{\section}{\large\bfseries\color{primary}\uppercase}{}{0pt}{}[\titlerule]
\titlespacing{\section}{0pt}{15pt}{8pt}

\hypersetup{
	colorlinks=true,      % Activa el color en el texto del link
	linkcolor=accent,     % Color para enlaces internos
	filecolor=accent,     % Color para archivos
	urlcolor=accent,      % Color para URLs externas (LinkedIn, etc)
	citecolor=accent,
	pdfborder={0 0 0}     % Elimina definitivamente cualquier recuadro
}


\begin{document}
	
	% TODO: Fix full height.
	% TODO: Adjust image size.
	% TODO: Move column to right for ATS
	
	% --- PINTAR EL FONDO DEL SIDEBAR
	% El segundo argumento [0pt][\paperheight] asegura que el color llegue al final del papel
	\backgroundcolor{c[0][0pt][\paperheight]}{sidebarGray}
	
	\begin{paracol}{2}
	
	% --- COLUMNA 0: SIDEBAR (IZQUIERDA) ---
	\vspace*{0.5cm} % Compensamos el margen superior que quitamos en geometry para el contenido
\hspace{0.02\textwidth} % Padding izquierdo respecto al separador de columnas
\begin{minipage}{0.90\textwidth} % Controla el margen derecho del texto
	
	% REDEFINICIÓN LOCAL PARA EL SIDEBAR (Sin líneas horizontales)
	\titleformat{\section}{\large\bfseries\color{primary}\uppercase}{}{0pt}{}
	\titlespacing{\section}{0pt}{12pt}{4pt}
	
	\includegraphics[]{sample.png}
	
	% --- LEYENDA DE ACTUALIZACIÓN ---
	\fontsize{7pt}{8pt}
	\textit{Actualizado: \today}
	
	\section*{Contacto}
	\small
	\textbf{Email}\\ \href{mailto:ramiro.md90@gmail.com}{ramiro.md90@gmail.com} \\[6pt]
	\textbf{Teléfono}\\ +54 221 123 4567 \\[6pt]
	\textbf{LinkedIn}\\ \href{https://linkedin.com/in/ramiro-martínez-d-elía-b65a9931}{/in/ramiro-martínez-d-elía-b65a9931}\\[6pt]
	\textbf{Github}\\ \href{https://github.com/ramiromd}{/ramiromd}	
	\section*{Datos personales}
	\small
	\textbf{Fecha de Nacimiento}\\29/06/1990 \\[6pt]
	\textbf{Número de documento (DNI)}\\ 35.609.454 \\[6pt]
	\textbf{Domicilio}\\ La Plata, Provincia de Buenos Aires,\\Argentina\\
	\section*{Idiomas}
	\small
	\textbf{Español}: Nativo \\
	\textbf{Inglés}: Lee y entiende \\
\end{minipage}

	
	% Comando mágico para saltar a la columna de la derecha
	\switchcolumn
	
	% --- COLUMNA 1: CONTENIDO PROFESIONAL (DERECHA) ---
	
		% --- COLUMNA 1: CONTENIDO PROFESIONAL (DERECHA) ---

% --- ENCABEZADO ---
% Compensamos el margen superior que quitamos en geometry para el contenido
\vspace*{0.5cm}
{\fontsize{34}{38}\selectfont \textbf{\color{primary} Ramiro}} \\[2pt]
{\fontsize{34}{38}\selectfont \textbf{\color{primary} Martínez D'Elía}} \\[8pt]
{\Large \textbf{\color{accent} Software Engineer}} \\

\section{Perfil Profesional}
\lipsum[1][1-4] % Párrafo largo de relleno

\section{Habilidades}
\setlength{\multicolsep}{0pt} % Quitamos el espacio superior de multicol
\begin{multicols}{2}
	\textbf{Backend \& Core}
	\begin{itemize}[noitemsep, topsep=2pt, leftmargin=10pt]
		\item Python (Django/FastAPI)
		\item Go (Microservicios)
		\item Node.js \& TypeScript
		\item Java (Spring Boot)
	\end{itemize}
	\columnbreak
	\textbf{Cloud \& DevOps}
	\begin{itemize}[noitemsep, topsep=2pt, leftmargin=10pt]
		\item AWS (Lambda, S3, EC2)
		\item Docker \& Kubernetes
		\item CI/CD (GitHub Actions)
		\item Terraform (IaC)
	\end{itemize}
\end{multicols}

\section{Educación}
\cvEducationLine{Analista Programador Universitario}{Completado}{Universidad Nacional de La Plata}


\section{Trayectoria}


% !TEX encoding = UTF-8

\cvExperienceHeader{Geopagos}{2020 -- Actualidad}{Ciudad Autónoma de Buenos Aires, Argentina}


\cvRoleProject{Product Owner Specialist}{Liquidación de impuestos y pagos a comercios}{2025 - Actualidad}

\begin{itemize}[itemsep=1pt, topsep=4pt, leftmargin=15pt]
	
	\item \textbf{Estrategia y gestión de backlog:} Lideré la priorización y refinamiento del backlog de producto y tecnología, logrando un equilibrio crítico entre los objetivos de negocio, la resolución de deuda técnica y la capacidad operativa del equipo.
	
	\item\textbf{ Diseño de roadmap y modernización arquitectónica:} Definí la visión estratégica para la evolución de la plataforma, diseñando una arquitectura orientada a servicios para el reemplazo progresivo de flujos legacy, priorizando métricas de \textbf{escalabilidad, resiliencia y mantenibilidad}.
	
	\item \textbf{Liderazgo de soluciones cross-team:} Coordiné la alineación de requerimientos complejos con líderes de producto y técnica de otros equipos, asegurando la consistencia funcional y técnica de los servicios transversales a nivel plataforma.
	
	\item \textbf{Gestión de partners y stakeholders:} Actué como enlace técnico y funcional principal con proveedores y socios externos, facilitando el refinamiento de requerimientos y liderando la evaluación, integración y consumo de servicios estratégicos del dominio.
	
	\item \textbf{Observabilidade}: Diseñé e implementé dashboards de control en herramientas de \textbf{APM}, estableciendo KPIs basados en métricas de performance y negocio (salud de flujos, volumen transaccional, percentiles y tiempos de respuesta) para el monitoreo proactivo de flujos críticos.
	
	\item \textbf{Análisis técnico funcional:} Aporté mi expertise como Technical Leader para validar la viabilidad de iniciativas, estimar esfuerzos y liderar el análisis de requerimientos no funcionales, incluyendo normativas impositivas internacionales, seguridad y performance.
	
\end{itemize}

\vspace{12pt} % Espacio entre proyectos dentro de la misma empresa

\cvRoleProject{Technical Lead}{Liquidación de impuestos y pagos a comercios}{2021 - 2022}

\begin{itemize}[itemsep=1pt, topsep=4pt, leftmargin=15pt]
	
	\item \textbf{Modernización y migración de arquitectura:} Lideré la transición tecnológica del ecosistema de liquidaciones, encabezando el diseño y construcción de una nueva arquitectura de microservicios basada en \textbf{Java} y \textbf{Spring Boot}. Este proceso permitió desacoplar flujos legacy críticos, mejorando la \textbf{escalabilidad} y \textbf{mantenibilidad} del sistema.
	
	
	\item \textbf{Refinamiento funcional y normativo:} Participé en el análisis estratégico de requerimientos asociados a normativas impositivas y de pagos a comercios multiregionales, traduciendo regulaciones de diversos países en especificaciones técnicas.
	
	\item \textbf{Gestión de riesgos y deuda técnica:} Lideré procesos de visibilización de deuda técnica mediante métricas de calidad y observabilidad, traduciendo hallazgos técnicos en riesgos de negocio comprensibles para \textbf{stakeholders}.
	
	
\end{itemize}

\vspace{12pt} % Espacio entre proyectos dentro de la misma empresa


\cvRoleProject{Senior backend developer}{Liquidación de impuestos y pagos a comercios}{2021 - 2022}

\begin{itemize}[itemsep=1pt, topsep=4pt, leftmargin=15pt]
	
	\item \textbf{Mantenimiento evolutivo de la solución:} Participación activa en el diseño, implementación y optimición de servicios backend críticos para la liquidación de impuestos y pagos a comercios, desarrollando motores de cálculo financiero tanto en tiempo real (\textbf{Online}) como para procesamiento masivo (\textbf{Batch}), asegurando la precisión, trazabilidad del flujo de fondos y el cumplimiento de calendarios de pago.
	
	\item \textbf{Liderazgo técnico y calidad de código:} Participé, junto con el Tech Lead, en los procesos de \textbf{Code Review} y \textbf{mentoría técnica} para desarrolladores del equipo, promoviendo estándares de diseño de software y el uso de mejores prácticas arquitectónicas.
	
\end{itemize}

\vspace{12pt} % Espacio entre proyectos dentro de la misma empresa
% !TEX encoding = UTF-8
\cvExperienceHeader{Gloobes}{2015 -- 2020}{La Plata, Buenos Aires, Argentina}

\cvRoleProject{Fullstack Developer}{Plataforma Ecommerce multiregional}{}
\begin{itemize}[itemsep=1pt, topsep=4pt, leftmargin=15pt]
	
	\item \textbf{Arquitectura de API REST:} Diseñé y desarrollé una API robusta para comercio electrónico utilizando \textbf{PHP 7} y \textbf{Phalcon 2}, gestionando flujos complejos de carrito de compras, procesamiento de pedidos y control de inventario.
	
	\item \textbf{Estrategia de Persistencia Híbrida:} Implementé una arquitectura de datos dual utilizando \textbf{MySQL} para la integridad de transacciones y \textbf{MongoDB} para el catálogo de productos, optimizando la flexibilidad y velocidad de respuesta en búsquedas de gran escala.
	
	\item \textbf{Integraciones Fintech Internacionales:} Realicé la integración de múltiples pasarelas de pago en \textbf{Argentina, Uruguay, México, Chile, Colombia, Estados Unidos, Portugal y Paraguay}, consumiendo servicios web mediante protocolos \textbf{SOAP} y \textbf{REST} para garantizar el pago de las ordenes de compra de los usuarios.
	
	\item \textbf{Logística:} Implementé la lógica de integración con diversos proveedores de envíos de \textbf{Argentina, Uruguay, México, Chile, Colombia, Estados Unidos, Portugal y Paraguay}, automatizando el cálculo de costos, tracking y logística mediante la conexión con sus respectivos \textbf{Web Services SOAP y/p REST}.
	
	\item \textbf{Escalabilidad Regional:} Adapté el sistema para manejar múltiples monedas, proveedores de pagos y proveedores logísticos locales, asegurando una experiencia de usuario consistente en diversos mercados.
	
	\item \textbf{Implementación de Frontend White Label:} Desarrollé un sistema de interfaz modular y personalizable para los distintos partners de la plataforma utilizando \textbf{PHP 7, Phalcon 2, JavaScript y CSS}, permitiendo el despliegue dinámico de múltiples tiendas bajo una misma base de código.
	
\end{itemize}

\vspace{10pt}

\cvRoleProject{Developer}{Emisión de comprobantes fiscales electrónicos}{}

\begin{itemize}
	\item \textbf{Desarrollo de una librería de facturación electrónica:} Diseñé e implementé una librería en \textbf{PHP 7} que funciona como una capa de abstracción para la integración con diversos proveedores de facturación electrónica. Esta solución estandarizó el envío de comprobantes fiscales mediante la normalización de datos para distintos Web Services gubernamentales y privados.
\end{itemize}

\vspace{10pt}

\cvRoleProject{Developer}{Integración de Punto de Venta Web (POS) y Hardware Fiscal}{}

\begin{itemize}[itemsep=1pt, topsep=4pt, leftmargin=15pt]
	
	\item \textbf{Desarrollo de Middleware de Comunicación:} Diseñé e implementé un servicio en \textbf{C\#} que actúa como capa de abstracción entre aplicaciones web y hardware local, permitiendo la comunicación de puntos de venta con impresoras fiscales mediante una \textbf{API de WebSockets}.
	\item \textbf{Integración de Librerías de Bajo Nivel:} Desarrollé un adaptador robusto para el consumo de librerías dinámicas (\textbf{DLL}) de diversos fabricantes de impresoras, estandarizando los comandos de impresión.
	\item \textbf{Estandarización de Protocolos:} Definí un protocolo de mensajes JSON sobre WebSockets para facilitar la integración de cualquier frontend web con periféricos locales, logrando una solución escalable y multiplataforma.
	
\end{itemize}

\vspace{10pt}

% !TEX encoding = UTF-8
\cvJobPosition{Full Stack Developer}{2016}{2016}{Freelogic, La Plata, Provincia de Buenos Aires, Argentina}{
	
	\cvProjectName{Un crucero}
	
	\begin{itemize}[itemsep=1pt, topsep=4pt, leftmargin=15pt]
		
		\item \textbf{Soporte evolutivo de la Interfaz de Usuario Web:} Participé en el desarrollo de mejoras continuas y nuevas funcionalidades para el \textbf{frontend web}, garantizando una respuesta fluida en las búsquedas y una experiencia de usuario intuitiva. Principalmente utilizando las tecnologías \textbf{Javascript} y \textbf{PHP 5} junto con \textbf{Symfony 2} para el \textbf{Backend For Frontend (BFF)}.
		
	\end{itemize}
}
\vspace{10pt}
% !TEX encoding = UTF-8
\cvExperienceHeader{Facultad de Ciencias Sociales y Jurídicas U.N.L.P. (Proyecto académico)}{2015 -- 2015}{La Plata, Buenos Aires, Argentina}

	
\begin{itemize}[itemsep=1pt, topsep=4pt, leftmargin=15pt]
	
	\item \textbf{Ingeniería de requerimientos (IEEE 830):} Participé de forma activa en la confección del documento \textbf{Software Requirements Specification (SRS)} bajo el estándar \textbf{IEEE 830}, documentando requerimientos funcionales y no funcionales para asegurar la trazabilidad y calidad del producto final.
	
	\item \textbf{Desarrollo de sistema de gestión:} Participé de forma activa en el desarrollo de una solución integral para la gestión de casos de los consultorios jurídicos gratuitos, ofrecidos por la Facultad, digitalizando el seguimiento de expedientes, agendas profesionales y archivos. Principalmente utilizando las tecnologías \textbf{PHP 5} y \textbf{Symfony 2}.
	
	\item \textbf{Modelado de datos}: Participé de forma activa en el modelado de la base de datos, garantizando la seguiridad y la confidencialidad de la información sensible de los consultantes.
	
\end{itemize}
\vspace{10pt}
% !TEX encoding = UTF-8
\cvJobPosition{Full Stack Developer}{2014}{2015}{Arzion S.R.L., La Plata, Buenos Aires, Argentina}{
	
	%Streame
	\cvProjectName{Streame}
	\begin{itemize}[itemsep=1pt, topsep=4pt, leftmargin=15pt]
		
		\item \textbf{Desarrollo de API REST:} Diseñé e implementé la arquitectura de la \textbf{API REST} utilizando \textbf{PHP 5} y \textbf{Phalcon Framework}, optimizando la entrega del catálogo global de estaciones de radio y garantizando tiempos de respuesta mínimos (baja latencia).
		
		\item \textbf{Gestión de Contenido Multimedia:} Desarrollé la lógica de backend para el procesamiento y entreega de enlaces de streaming en tiempo real, permitiendo la reproducción fluida de contenido de emisoras de todo el mundo.
		
		\item \textbf{Estandarización:} Estructuré el catálogo de radios mediante un sistema de base de datos optimizado, facilitando la búsqueda, filtrado y consumo de metadatos por parte del frontend y aplicaciones móviles.
	\end{itemize}
	
	% Trip Tuner
	\cvProjectName{Trip Tuner}
	\begin{itemize}[itemsep=1pt, topsep=4pt, leftmargin=15pt]
		
		\item \textbf{Soporte evolutivo de Motor de recomendaciones:} Participé en el desarrollo de mejoras continuas y nuevas funcionalidades para el sistema interactivo de descubrimiento de destinos turísticos, traduciendo las entradas del usuario en consultas dinámicas de base de datos para ofrecer resultados personalizados. Principalmente utilizando las tecnologías \textbf{PHP 5} y \textbf{Codeigniter}.
		
		\item \textbf{Soporte evolutivo de la Interfaz de Usuario Web:} Participé en el desarrollo de mejoras continuas y nuevas funcionalidades para el \textbf{frontend web}, garantizando una respuesta fluida en las búsquedas y una experiencia de usuario intuitiva. Principalmente utilizando las tecnologías \textbf{Javascript}.
		
		\item \textbf{Optimización de performance de búsquedas:} Participé en el \textbf{análisis de las consultas de búsqueda, la creación estratégica de índices y la implementación de mecanismos de caché}, en función de recuperar los resultados de forma eficiente bajo múltiples criterios de filtrado. Principalmente utilizando las tecnologías \textbf{MySQL} y \textbf{Redis}.
		
		\item \textbf{Adaptabilidad (White Label):} Participé en el diseño e implementación de adaptaciones de la plataforma para su despliegue en \textbf{landings de partners estratégicos}, permitiendo replicar la funcionalidad de recomendación bajo diferentes indentidades de marca y configuraciones. \textbf{PHP 5}, \textbf{Codeigniter}, \textbf{Javascript} y \textbf{CSS}.
	\end{itemize}
	
	% Regatta Travel Solutions
	\cvProjectName{Regatta Travel Solutions}
	\begin{itemize}[itemsep=1pt, topsep=4pt, leftmargin=15pt]
		
		\item \textbf{Soporte evolutivo de Landing Pages:} Participé en el desarrollo de mejoras continuas y personalizaciones en las landing pages de los \textit{partners} estrategicos de la plataforma. Principalmente utilizando las tecnologías \textbf{PHP 5}, \textbf{Codeigniter}, \textbf{Javascript} y \textbf{CSS}.
		
		\item \textbf{Desarrollo evolutivo de Backoffice de inventario hotelero:} Participé en el desarrollo de mejoras y nuevas funcionalidades en el panel administrativo utilizado por los hoteles. Principalmente utilizando las tecnologías \textbf{PHP 5}, \textbf{Codeigniter}.
	\end{itemize}
}
\vspace{10pt}
% !TEX encoding = UTF-8
\cvExperienceHeader{Colegio Manantiales}{2011 -- 2014}{La Plata, Buenos Aires, Argentina}
\begin{itemize}[itemsep=1pt, topsep=4pt, leftmargin=15pt]
	\item \textbf{Desarrollo de Aplicaciones Web:} Diseñé e implementé un ecosistema digital compuesto por un \textbf{Backoffice administrativo} y un \textbf{Portal de autogestión para tutores}, ambos construidos con \textbf{PHP 5} y \textbf{Symfony 2}.
	
	\item \textbf{Gestión de Identidades y Roles:} Implementé un sistema de autenticación y control de acceso basado en roles para la administración de catálogos críticos (alumnos, padres y personal), garantizando la seguridad y privacidad de los datos.
	
	\item \textbf{Optimización de Recursos y Logística:} Desarrollé un módulo de selección de menús diarios que permitió la proyección de demanda de alimentos, logrando una \textbf{optimización del inventario} y una reducción significativa en el desperdicio de insumos.
	
	\item \textbf{Optimización de costos y comunicación digital}: Desarrollé un módulo de \textbf{comunicaciones institucionales} en \textbf{PHP 5} y \textbf{Symfony 2} que reemplazo las notificaciones papel, logrando una reducción en los costos operativos y de suministros.
	
	\item \textbf{Aplicación para personal de cocina:} Programé una interfaz de usuario avanzada para el área de cocina utilizando \textbf{ExtJS 5}, permitiendo el procesamiento y visualización de pedidos en tiempo real para agilizar las compras y la preparación de comidas.
	
\end{itemize}
\vspace{10pt}


% TODO: Didn't work
% Forzamos que la página se mantenga abierta hasta el final
%\vspace*{\fill}\null
%\miFirma
%\input{example}
	\end{paracol}
	
	
\end{document}